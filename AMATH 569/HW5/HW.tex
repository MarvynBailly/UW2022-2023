\documentclass[12pt]{report}

\usepackage{commands}


\begin{document}

\large

\begin{center}
 Math 569 Homework 5\\
 Due May 24\\
 By Marvyn Bailly\\
\end{center}

\normalsize

\hrule

%---------------%
%---Problem 1---%
%---------------%

%--status--$

\begin{problem}
    For the 1-dimensional heat equation for conduction in a copper rod:
    \[
        \begin{cases}
            \pp{}{t}u = \alpha^2 \ppn{2}{}{x}u, &0<x<L,t>0\\
            u(x,t) = 0, &\text{at}~x=0 \and x=L,\\
            u(x,0) = f(x), &0 < x < L.
        \end{cases}
    \] 
    \begin{enumerate}
        \item [(a)]
        solve using separation of variables. Show that the time-dependence for the nth standing mode is $\text{exp}(-n^2(t/t_\eps))$ where $t_\eps = (L/\pi \alpha)^2$, about 1 hour for a 2m-copper rod.


        \item [(b)]
        For $t>t_\eps$, find the mode that dominates the solution, and thus write down the
        approximate solution (in time and in space). Describe in words how an initial condition,
        which may not look like a sine wave, becomes the sine shape with the largest wavelength
        fitting in between the boundaries.

    
    \end{enumerate}
\end{problem}

\begin{solution}
    
    \noindent
    Consider the 1-dimensional heat equation for conduction in a copper rod:
    \begin{equation}\label{1}
        \begin{cases}
            \pp{}{t}u = \alpha^2 \ppn{2}{}{x}u, &0<x<L,t>0\\
            u(x,t) = 0, &\text{at}~x=0 \and x=L,\\
            u(x,0) = f(x), &0 < x < L.
        \end{cases}
    \end{equation}
    \begin{enumerate}
        \item [(a)]
        
        \noindent
        First, we wish to solve \fullref{1} using separation of variables. Consider
        \[
            u(x,t) = X(x)T(t).
        \] 
        Plugging this into \fullref{1} and rearranging terms we find that
        \[
            \frac{T'(t)}{T(t)\alpha^2} = \frac{X''(x)}{X(x)} = - \lambda^2, 
        \]
        where $\lambda$ is the separation constant. Thus we have two ODEs of the form
        \[
            T'(t) + \lambda^2\alpha^2T(t) = 0,
        \] 
        and
        \[
            X''(x) + \lambda^2 X(x) = 0.
        \]
        We will first solve the former of the ODEs to get
        \[
            T(t) = T(0)e^{-\alpha^2\lambda^2t}.
        \]
        Recalling that $u(0,t) = u(L,t) = 0$ which implies $X(0) = X(L) = 0$, we can solve the latter ODE to get
        \[
            X(x) = X_n(x) = \sin\paren{\frac{n \pi x}{L}},
        \] 
        where $X_n$ are eigenfunctions with corresponding eigenvalues
        \[
            \lambda = \lambda_n  = \frac{n\pi}{L},
        \] 
        for $n = 1,2,\dots$. Thus we have found
        \[
            u(x,t) = \sum_{n=1}^{\infty} T_n(0)e^{-\alpha^2\paren{\frac{n\pi}{L}}^2t} \sin\paren{\frac{n \pi x}{L}}. 
        \]
        If we let $t_\eps = \paren{\frac{L}{\pi \alpha}}^2$ then 
        \[
            u(x,t) = \sum_{n=1}^{\infty} f_n(x) e^{\frac{-n^2 t}{t_\eps}} \sin\paren{\frac{n \pi x}{L}},
        \]
        where we use that $u(x,0) = f(x)$ which requires
        \[
            f(x) = \sum_{n=1}^\infty T_n(0) \sin\paren{\frac{n \pi x}{L}},
        \]
        and using a Fourier sine series we get
        \[
            f(x) = \sum_{n=1}^\infty f_n \sin\paren{\frac{n \pi x}{L}}.
        \]
        Thus we see that the time-dependence for the $n$th standing mode is $\text{exp}(-n^2(t/t_\eps))$.


        \item [(b)]
        For $t > t_\eps$, we can approximate the solution by finding the dominant mode. Recall that the solution is
        \[
            u(x,t) = \sum_{n=1}^{\infty} f_n(x) e^{\frac{-n^2 t}{t_\eps}} \sin\paren{\frac{n \pi x}{L}}
        \] 
        which has terms with exponential decay. Thus to find the mode which will die the slowest and dominate the other modes, we must minimize $n$. Therefore the dominate mode is $n=1$ and so
        \[
            u(x,t) \approx f_1 e^{-t/t_\eps}\sin\paren{\frac{\pi x}{L}}.
        \]
        Furthermore, this tells us that any initial condition for \fullref{1}, may it be sin-like or not, will tend towards the above approximation which has a sine shape with the largest wavelength fitting in between the boundaries. 


    \end{enumerate}
\end{solution}

%----------------------------------------------------------------------------------------------------%
%\vskip 20pt
\newpage

%---------------%
%---Problem 1---%
%---------------%

%--status--$

\begin{problem}
    Sound waves in a box satisfies
    \[
        \begin{cases}
            \ppn{2}{}{t}u - c^2\nabla^2u = 0, &\text{in} ~ V\\
            u = 0, &\text{on} ~ \partial V.
        \end{cases}
    \] 
    Use separation of variables to solve this problem for the following configurations and find the quantized frequency of oscillation $\omega$, where $\omega$ appears in the time dependence of the solution in the form of a sine or cosine of $\omega t$.
    \begin{enumerate}
        \item [(a)] $V$ is a one-dimensional box: $0<x<L$.
        \item [(b)] $V$ is a two-dimensional box: $0<x<L; 0<y<L$.
        \item [(c)] $V$ is a three-dimensional box: $0<x<L; 0<y<L; 0<z<L$.
    \end{enumerate}
\end{problem}

\begin{solution}

    \noindent
    Consider sound waves in a box which satisfy
    \begin{equation}\label{2}
        \begin{cases}
            \ppn{2}{}{t}u - c^2\nabla^2u = 0, &\text{in} ~ V\\
            u = 0, &\text{on} ~ \partial V.
        \end{cases}
    \end{equation}
    We wish to solve \fullref{2} using the separation of variables method. Consider
    \[
        u(x,t) = X(x)T(t).
    \] 
    Plugging this into \fullref{2} and rearranging terms we find that
    \[
        \frac{T''(t)}{T(t)c^2} = \frac{\nabla^2 X(x)}{X(x)} = - \lambda^2, 
    \]
    where $\lambda$ is the separation constant. Thus we have two ODEs of the form
    \[
        T''(t) + \lambda^2 c T(t) = 0,
    \] 
    and 
    \begin{equation} \label{2-x}
        \nabla^2 X(x) + \lambda^2 X(x) = 0.
    \end{equation}
    Notice that in any domain $V$, we can solve the former ODE to get 
    \begin{equation} \label{2-t-sol}
        T(t) = A\sin(c \lambda t) + B\cos(c \lambda t).
    \end{equation}

    \begin{enumerate}
        \item [(a)]
        Let $V$ be a one-dimensional box given by 
        \[
            0 < x < L.
        \]    
        Thus \fullref{2-x} becomes
        \[
            X''(x) + \lambda^2 X(x) = 0.
        \]
        Since $u = 0$ on $\partial V$, or in other words, $u(0,t) = u(L,t) = 0$ which implies that $X(0) = X(L) = 0.$  This has eigenfunctions
        \[
            X(x) = X_n(x) = \sin\paren{\frac{n\pi x}{L}},
        \]
        with corresponding eigenvalues
        \[
            \lambda = \lambda_n = \frac{n \pi}{L},
        \]
        for $n=1,2,\dots$. Thus in combination with \fullref{2-t-sol} we have
        \[
            u(x,t) = \sum_{n=1}^\infty \paren{A\sin\paren{\frac{c n \pi t}{L}} + B\cos\paren{\frac{c n \pi t}{L}}}\sin\paren{\frac{n\pi x}{L}}.
        \]
        Therefore the quantized frequency of oscillation $\omega$ is given by
        \[
            \omega_n = c \lambda_n = \frac{c n \pi}{L},
        \]
        for $n=1,2,\dots$.

        \item [(b)]
        Let $V$ be a two-dimensional box given by 
        \[
            0 < x < L; 0 < y < L.
        \]
        Thus \fullref{2-x} becomes
        \[
            X''(\vec{x}) + \lambda^2 X(\vec{x}) = 0,
        \]
        where $\vec{x} = (x,y)^T$. Since $u = 0$ on $\partial V$, or in other words, $u(0,t) = u(L,t) = 0$ which implies that $X(\vec{x}) = 0$ when $x=0,x=L,y=0,\and y=L$. This has eigenfunctions
        \[
            X(\vec{x}) = X_{n,m}(\vec{x}) = \sin\paren{\frac{n\pi x}{L}}\sin\paren{\frac{m\pi y}{L}},
        \]
        with corresponding eigenvalues
        \[
            \lambda = \lambda_{n,m} = \frac{\pi\sqrt{n^2 + m^2}}{L},
        \]
        for $n,m=1,2,\dots$. Thus in combination with \fullref{2-t-sol} we have
        \[
            u(x,t) = \sum_{n,m=1}^\infty \paren{A\sin\paren{\frac{c\pi\sqrt{n^2 + m^2} t}{L}} + B\cos\paren{\frac{c\pi\sqrt{n^2 + m^2} t}{L}}}\sin\paren{\frac{n\pi x}{L}}\sin\paren{\frac{m\pi y}{L}}.
        \]
        Therefore the quantized frequency of oscillation $\omega$ is given by
        \[
            \omega_{n,m} = c \lambda_{n,m} = \frac{c \pi \sqrt{n^2 + m^2}}{L},
        \]
        for $n,m=1,2,\dots$.

        \item [(c)]
        Let $V$ be a three-dimensional box given by 
        \[
            0 < x < L; 0 < y < L; 0 < z < L.
        \]
        Thus \fullref{2-x} becomes
        \[
            X''(\vec{x}) + \lambda^2 X(\vec{x}) = 0,
        \]
        where $\vec{x} = (x,y,z)^T$. Since $u = 0$ on $\partial V$, or in other words, $u(0,t) = u(L,t) = 0$ which implies that $X(\vec{x}) = 0$ when $x=0,x=L,y=0,y=L,z=0,\and z=L$. This has eigenfunctions
        \[
            X(\vec{x}) = X_{n,m,k}(\vec{x}) = \sin\paren{\frac{n\pi x}{L}}\sin\paren{\frac{m\pi y}{L}}\sin\paren{\frac{k \pi z}{L}},
        \]
        with corresponding eigenvalues
        \[
            \lambda = \lambda_{n,m,k} = \frac{\pi\sqrt{n^2 + m^2 + k^2}}{L},
        \]
        for $n,m,k=1,2,\dots$. Thus in combination with \fullref{2-t-sol} we have
        \begin{align*}
            u(x,t) &= \sum_{n,m,k=1}^\infty \paren{A\sin\paren{\frac{c\pi\sqrt{n^2 + m^2 + k^2} t}{L}} + B\cos\paren{\frac{c\pi\sqrt{n^2 + m^2 + k^2} t}{L}}}\\
            &\quad \sin\paren{\frac{n\pi x}{L}}\sin\paren{\frac{m\pi y}{L}}\sin\paren{\frac{k\pi y}{L}}.    
        \end{align*}
        
        Therefore the quantized frequency of oscillation $\omega$ is given by
        \[
            \omega_{n,m,k} = c \lambda_{n,m,k} = \frac{c \pi \sqrt{n^2 + m^2 + k^2}}{L},
        \]
        for $n,m,k=1,2,\dots$.
    
    \end{enumerate}






\end{solution}

%----------------------------------------------------------------------------------------------------%
%\vskip 20pt
\newpage

%---------------%
%---Problem 1---%
%---------------%

%--status--$

\begin{problem}
    \begin{enumerate}
        \item [(a)]
        Solve:
        \[
            \ddn{2}{}{x}u + \paren{k_0^2 + i\eps k_0 / c}u = -\delta(x -y)/c^2,
        \]
        where $-\infty < x < \infty$, $\eps > 0$, and $y$ is finite. subject to the boundary condition $u\to 0$ as $x \to \pm \infty$. Consider separately $x<y$ and $x>y$ and matching across $x=y$. Do not use Fourier transform.
        
        
        
        \item [(b)]
        Solve the above equation for the case of $\eps = 0$, subject to the Sommerfeld radiation condition. Show that the solution from (a) reduce to this solution a $\eps \to 0$.
        
        
    \end{enumerate}
    


\end{problem}

\begin{solution}

    \noindent
    Consider the equation
    \begin{equation}\label{3}
        \ddn{2}{}{x}u + \paren{k_0^2 + \frac{i\eps k_0}{c}}u = -\frac{\delta(x -y)}{c^2},
    \end{equation}
    
    where $-\infty < x < \infty$ and $y$ is finite and is subject to the boundary condition $u \to 0$ as $x \to \pm \infty$.

    \begin{enumerate}
        \item [(a)]
        We first consider \fullref{3} when $\eps > 0$. Let's begin by considering the case when $x < y$ giving 
        \[
            \ddn{2}{}{x}u + \paren{k_0^2 + \frac{i\eps k_0}{c}}u = 0,
        \]      
        with $-\infty < x < y$. We note that the roots of the characteristic polynomial are
        \[
            \lambda_{1,2} = \pm \sqrt{-\paren{k_0^2 + \frac{i \eps k_0}{c}}}.
        \]
        Since $\lambda_{1,2}$ contains a square root term we require a branch cut. Consider the principal branch of the square root with the branch cut $(-\infty,0]$.
        Assuming that $k_0, c> 0$ and letting
        \[
            \lambda_2 = \sqrt{-\paren{k_0^2 + \frac{i \eps k_0}{c}}},
        \]
        which has a negative real part and a positive imaginary part. We also note that $\lambda_1 = -\lambda_2$. Thus the solution is
        \[
            u(x) = A_1 e^{\lambda_1 x} + B_1 e^{\lambda_2 x}.
        \] 
        To enforce the boundary condition that $u \to 0$ as $x \to -\infty$, we require that $B_1 = 0$ and thus the solution is
        \[
            u(x) = A_1 e^{\lambda_1 x} = A e^{\lambda_1 x}.
        \]
        In the case that $x > y$, we have 
        \[
            \ddn{2}{}{x}u + \paren{k_0^2 + \frac{i\eps k_0}{c}}u = 0,
        \]
        where $y < x < \infty$. We can solve this similarly to before to get
        \[
            u(x) = A_2 e^{\lambda_1 x} + B_2 e^{\lambda_2 x}.
        \] 
        To enforce the boundary condition that $u \to 0$ as $x \to \infty$, we require that $A_2 = 0$ and thus the solution is
        \[
            u(x) = B_2e^{\lambda_2 x} = Be^{\lambda_2 x}.
        \]
        Combining the cases gives 
        \begin{equation}\label{3-intso}
            u(x) = \begin{cases}
                A e^{\lambda_1 x}, & x<y,\\
                B e^{\lambda_2 x}, & x>y.\\
            \end{cases}
        \end{equation}
        
        Now to solve for $A$ and $B$ we will enforce a matching condition across $x = y$. Recalling that $\lambda_1 = -\lambda_2$ we get that 
        \[
            A e^{\lambda_1 y} = B e^{-\lambda_1 y} \implies B = A e^{2\lambda_1 y}.
        \]  
        Now letting $y^-$ and $y^+$ be arbitrarily close to $x = y$ on their repsective sides, we integrating across $x=y$ to get 
        \[
            \int_{y^-}^{y^+} u_{xx}\d x +  k^2\int_{y^-}^{y^+} u(x) \d x = -\int_{y^-}^{y^+} \frac{\delta(x -y)}{c^2} \d x = \frac{-1}{c^2},
        \] 
        and assuming that $u$ is finte we get
        \[
            \int_{y^-}^{y^+} u_{xx}\d x = u_x(y^+) - u_x(y^-) = - \frac{1}{c^2}.
        \]
        Using \fullref{3-intso} we derive
        \[
            Ae^{2\lambda_1 y}(-\lambda_1)\paren{e^{-\lambda_1 y}} - A \lambda_1e^{\lambda_1 y} = - \frac{1}{c^2} \implies 2A \lambda_1 e^{\lambda_1 y} = \frac{1}{c^2}. 
        \]
        Thus we have found 
        \[
            A = \frac{1}{2c^2\lambda_1}e^{-\lambda_1 y},
        \]
        and 
        \[
            B = \frac{1}{2c^2\lambda_1}e^{\lambda_1 y}.
        \]
        Therefore the solution is
        \[
            u(x) = \begin{cases}
                \frac{1}{2c^2\sqrt{-\paren{k_0^2 + \frac{i\eps k_0}{c}}}}e^{\sqrt{-\paren{k_0^2 + \frac{i\eps k_0}{c}}}(y-x)}, &x<y,\\
                \frac{1}{2c^2\sqrt{-\paren{k_0^2 + \frac{i\eps k_0}{c}}}}e^{\sqrt{-\paren{k_0^2 + \frac{i\eps k_0}{c}}}(x-y)}, &x>y.\\
            \end{cases}
        \]


        \item [(b)]
        We now consider \fullref{3} with $\eps = 0$ subject to the Sommerfeld radiation condition. We again consider the case when $x<y$ which gives
        \[
            \ddn{2}{}{x}u + \paren{k_0^2 + \frac{i\eps k_0}{c}}u = 0,
        \]      
        with $-\infty < x < y$. This has the solution
        \[
            u(x) = A_1 e^{ik_0 x} + B_1 e^{-ik_0 x}.
        \]
        Now we enforce the Sommerfeld radiation condition and assuming that $k_0 > 0$ we see that $A_1 = 0$. We again consider the case when $x>y$ which gives
        \[
            \ddn{2}{}{x}u + \paren{k_0^2 + \frac{i\eps k_0}{c}}u = 0,
        \]      
        with $y > x > \infty$. This has the solution
        \[
            u(x) = A_2 e^{ik_0 x} + B_2 e^{-ik_0 x}.
        \]
        Now we enforce the Sommerfeld radiation condition and assuming that $k_0 > 0$ we see that $B_2 = 0$. Thus we have
        \[
            u(x) = \begin{cases}
                B e^{-ik_0 x}, &x < y,\\
                A e^{ik_0 x}, &x > y.
            \end{cases}
        \] 
        Now to solve for $A$ and $B$ we will enforce a matching condition across $x = y$. Thus we have that 
        \[
            B e^{-i k_0 y} = A e^{i k_0 y} \implies A = B e^{-2 i k_0 y}.
        \]  
        Now letting $y^-$ and $y^+$ be arbitrarily close to $x = y$ on their repsective sides, we integrating across $x=y$ to get 
        \[
            \int_{y^-}^{y^+} u_{xx}\d x +  k^2\int_{y^-}^{y^+} u(x) \d x = -\int_{y^-}^{y^+} \frac{\delta(x -y)}{c^2} \d x = \frac{-1}{c^2},
        \] 
        and assuming that $u$ is finte we get
        \[
            \int_{y^-}^{y^+} u_{xx}\d x = u_x(y^+) - u_x(y^-) = - \frac{1}{c^2}.
        \]
        Using \fullref{3-intso} we derive   
        \[
            \paren{B e^{-2ik_0 y}}(ik_0)e^{i k_0 y} - B(-ik_0)e^{-ik_0y} = - \frac{1}{c^2},
        \]
        and solving gives
        \[
            B = \frac{-1}{2c^2 i k_0}e^{ik_0 y},
        \]
        and
        \[
            A = \frac{-1}{2c^2 i k_0}e^{-ik_0 y}.
        \]
        Therefore the solution is
        \[
            u(x) = \begin{cases}
                \frac{i}{2c^2 k_0}e^{ik_0(y-x)}, & x < y,\\ 
                \frac{i}{2c^2 k_0}e^{ik_0(x-y)}, & x > y.\\ 
            \end{cases}
        \] 
        
        \noindent
        To see that this solution is consistent with the solution from part a, we take $\eps \to 0^+$ and recalling our branch we get
        \[
            \lim_{\eps \to 0^+} \sqrt{-\paren{k_0^2 + \frac{i \eps k_0}{c}}} = \sqrt{-k^2_0} = ik_0,
        \]
        Thus when $\lambda_1 = -ik_0$ we get
        \[
            u(x) = \begin{cases}
                \frac{1}{2c^2k_0}e^{ik_0(y-x)}, & x < y,\\ 
                \frac{1}{2c^2k_0}e^{ik_0(x-y)}, & x > y,\\ 
            \end{cases}
        \] 
        which matches the solution we found.
    \end{enumerate}
\end{solution}

%----------------------------------------------------------------------------------------------------%
%\vskip 20pt
\newpage

\end{document}