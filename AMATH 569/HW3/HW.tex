\documentclass[12pt]{report}

\usepackage{commands}
\usepackage{cancel}
\usepackage{tikz}

\begin{document}

\large

\begin{center}
 Math 569 Homework 3\\
 Due 3 May\\
 By Marvyn Bailly\\
\end{center}

\normalsize

\hrule

%---------------%
%---Problem 1---%
%---------------%

%--status--$

\begin{problem}
    \begin{enumerate}
        \item [(a)]
        Solve using Fourier transform in $x$ and Laplace transform in $t$:
        \begin{align*}
            \begin{cases} 
                \pp{u}{t} - D \ppn{2}{u}{x} = \delta(x - \xi)\delta(t - \tau),  &-\infty < x < \infty, t > 0, -\infty < \xi < \infty, \tau > 0,\\
                u(x,t) \to 0, &\text{as}~ x \to \pm \infty, t>0,\\
                u(x,0) = 0, &\infty<x<\infty. 
            \end{cases}
        \end{align*}
        \item [(b)]
        Solve the same problem as in (a) expect you do not use Laplace transform in $t$. 


    \end{enumerate}
    
\end{problem}

\begin{solution}
    \def\x{{\xi}}
    \def\t{{\tau}}
    \def\w{{\omega}}

    \noindent
    Consider the equation
    \begin{align} \label{big boi}
        \begin{cases} 
            \pp{u}{t} - D \ppn{2}{u}{x} = \delta(x - \xi)\delta(t - \tau),  &-\infty < x < \infty, t > 0, -\infty < \xi < \infty, \tau > 0,\\
            u(x,t) \to 0, &\text{as} x \to \pm \infty, t>0,\\
            u(x,0) = 0, &\infty<x<\infty. 
        \end{cases}
    \end{align}
    \begin{enumerate}
        \item[(a)]
        We wish to solve \ref{big boi} using a Fourier transform in $x$ and Laplace transform in $t$. Let's first take the Fourier transform in $x$. We first define
        \[ 
            U(\w,y) = \F[u(x,y)] = \int_{-\infty}^{\infty}u(x,y)e^{i\w x}dx.
        \]  
        Then using integration by parts twice we find that,
        \begin{align*}
            \F[u_{xx}] &= \int_{-\infty}^{\infty}u_{xx}e^{i\w x}\d x\\
            &= [u_x e^{i\w x}]_{-\infty}^{\infty} - i\w \int_{-\infty}^{\infty} u_x e^{i\w x}\d x\\
            &= \cancel{[u_x e^{i\w x}]_{-\infty}^{\infty}} - i\w \cancel{[ue^{i\w x}]_{-\infty}^\infty} - \w^2 \int_{-\infty}^{\infty} u e^{i \w x}\d x\\
            &= -\w^2 U,
        \end{align*}
        where the first term is canceled as $u(x,y) \to 0$ as $x \to \pm \infty$ and the second term is canceled by making the assumption that $u_x \to 0$ as $|x| \to \infty$. We also compute
        \[ 
            \F[u_{t}] = \int_{-\infty}^{\infty} u_{t}e^{i\w x}\d x = \ppn{2}{}{y}\int_{-\infty}^{\infty}ue^{i\w x} \d x = U_{t},
        \]
        and 
        \begin{align*}
            \F[\delta(x - \x)\delta(t - \t)] &= \int_{-\infty}^{\infty}\delta(x - \x)\delta(t - \t)e^{i \w x} \d x\\ 
            &= \delta(t - \t)\int_{-\infty}^{\infty}\delta(x - \x)e^{i \w x} \d x\\
            &= \delta(t - \t) e^{i\w\x}.
        \end{align*}
        Thus we can rewrite equation \ref{big boi} as
        \[ 
            U_t + D \w^2 U = \delta(t - \t) e^{i\w\x}.
        \]
        Next we will take Laplace transform in $t$. We first define
        \[ 
            \tilde{U}(\w, s) = \L[U(\w,t)].
        \]
        Then we have
        \begin{align*}
            \L[U(\w,t)_t] &= \int_0^\infty U_t e^{-st} \d t\\
            &= \cancel{[U e^{-st}]_0^\infty} + s \int_{0}^{\infty} U e^{-st} \d t\\
            &= s \tilde{U}(\w, s),
        \end{align*}
        if we make the assumption that $u(x,t) \to 0$ as $t \to \infty$. We also have
        \[ 
            \L[\delta(t - \t) e^{i\w\x}] = \int_0^\infty \delta(t - \t) e^{i\w\x} e^{-st} \d t = e^{i\w\x} \int_0^\infty \delta(t - \t)e^{-st} \d t = e^{i\w\x}e^{-s \t}. 
        \]
        Thus the PDE becomes
        \[ 
            (s + D\w^2)\tilde{U}(\w,s) = e^{i\w\x}e^{-s \t} = e^{i\w\x -s\t},
        \]
        and so
        \[ 
            \tilde{U}(\w,s) = \frac{e^{i\w\x-s \t}}{s + D\w^2}.  
        \]
        Now we will compute the inverse Laplace transform
        \[ 
            U(\w,t) = \L^{-1}[\tilde{U}(\w,s)] = \frac{1}{2\pi i }\int_{\alpha-i\infty}^{\alpha+i\infty} e^{st}\frac{e^{i\w\x-s \t}}{s + D\w^2} \d s = \frac{e^{i\w\xi}}{2\pi i}\int_{\alpha-i\infty}^{\alpha + i\infty}  \frac{e^{s(t-\tau)}}{s + D\w^2} \d s. 
        \]
        \begin{figure}[H]
            \center
            \begin{tikzpicture}[x=1cm,y=1cm]
                % Draw axes
                \draw[->] (-5,0) -- (5,0) node[right] {Re};
                \draw[->] (0,-5) -- (0,5) node[above] {Im};
                % Draw line for alpha
                \draw[dashed] (1,4) -- (1,-4) node[below] {$\alpha$};
    
                % Draw right semicircle
                \draw[thick,red] (1,4) arc (90:-90:4);
    
                
                % Draw left semicircle
                \draw[thick,blue] (1,4) arc (90:270:4); %node[left] {$C_3$};
    
    
                % Draw straight line
                %\draw[thick,red] (1,4) -- (1,-4);
    
                \filldraw[black] (-1,0) circle (2pt) node[anchor=north] {$s=-D\w^2$};
            \end{tikzpicture}
            \caption{Bromwich Contour}
        \end{figure}
        Note that there exists a simple pole at $s = - D\w^2$. To compute the integral, we first consider when $t < \tau$ which gives exponential decay when $s > 0$. Thus we use a Bromwich contour, $C^+$ on the right side centered at $\alpha$ (up along the dotted line and down along the red arc). Then we have that
        \[ 
            \int_{C^+}  \frac{e^{s(t - \tau)}}{s + D\w^2} \d s = 0,
        \]
        by Jordan's lemma since the integrand is analytic for $s>0$. When $t > \tau$, we have exponential decay when $s < 0$ and thus we will use a Bromwich contour, $C^-$ on the left side centered at $\alpha$ (down along the dotted line and up along the blue arc). We note that $C^-$ now contains the simple pole. Then using Jordan's lemma we have
        \begin{align*}
            \int_{C^-}  \frac{e^{s(t - \tau)}}{s + D\w^2} \d s &= 2\pi i \sum \text{Res} \paren{\frac{e^{s(t - \tau)}}{s + D\w^2}}\\
            &= 2 \pi i \frac{e^{(-D\w^2)(t - \tau)}}{(s + D \w^2)'}\\
            &= 2 \pi i e^{-D\w^2(t - \tau)}.
        \end{align*}
        Thus we have found
        \[ 
            U(\w,t) = \begin{cases}
                0, \quad t< \tau\\
                e^{i\w\x}e^{-D\w^2(t - \tau)},\quad t > \tau
            \end{cases},
        \]
        and if we let $H(x)$ denote the Heaviside function we can write this as
        \[ 
            U(\w,t) = H(t-\tau)e^{i\w\x - D\w^2(t - \tau)},
        \] 
        which is the solution in the frequency domain. Finally we have to take the inverse Fourier transform
        \begin{align*}
            u(x,t) = \F^{-1}[U(\w,t)] &= \frac{1}{2\pi} \int_{-\infty}^{\infty} e^{ - i x \w} e^{i\xi \w} H(t-\tau)e^{i\w\x - D\w^2(t - \tau)} \d\w\\
            &= \frac{H(t-\tau)}{2\pi} \int_{-\infty}^{\infty} e^{-i\w(x - \x) + D\w^2(t - \t)} \d\w\\
            &= \frac{H(t-\tau)}{2\pi} \int_{-\infty}^{\infty} e^{-D(t - \t)\paren{\w^2 + \frac{i \w(x - \x)}{D(t-\t)}}} \d\w\\
            &= \frac{H(t-\tau)}{2\pi} \int_{-\infty}^{\infty} e^{-D(t - \t)\paren{\paren{\w + \frac{i (x - \x)}{2D(t-\t)}}^2 + \paren{\frac{x-\x}{2D(t- \t)}}^2}} \d\w\\
            &= \frac{H(t-\tau)}{2\pi}e^{-\paren{\frac{x-\x}{2D(t- \t)}}^2} \int_{-\infty}^{\infty} e^{-D(t - \t)\paren{\w + \frac{i (x - \x)}{2D(t-\t)}}^2} \d\w.\\
        \end{align*}
        Recall that the definite integral of a Gaussian function is given by 
        \[ 
            \int_{-\infty}^{\infty}e^{-a(x+b)^2} = \sqrt{\frac{\pi}{a}},
        \]
        and thus we can solve to the integral to be
        \[ 
            \int_{-\infty}^{\infty} e^{-D(t - \t)\paren{\w + \frac{i (x - \x)}{2D(t-\t)}}^2} \d\w = \sqrt{\frac{\pi}{D(t-\t)}}. 
        \]
        Therefore we have found the solution to be
        \[
            u(x,t) = \frac{H(t-\tau)}{2\pi}e^{-\paren{\frac{x-\x}{2D(t- \t)}}^2}\sqrt{\frac{\pi}{D(t-\t)}} = \frac{H(t - \tau)e^{-\paren{\frac{x-\x}{2D(t- \t)}}^2}}{\sqrt{4\pi D(t - \t)}},
        \]
        where the Heaviside function covers the case when $t < \t$. Finally we can check the assumptions we made. Clearly we have that $u_x \to 0$ as $x \to \pm \infty$ do to the exponential term and $u \to 0$ as $t \to \infty$ since the exponential terms goes to $1$ and the denominator grows large. 

        \item[(b)]
        Now we wish to solve equation \ref{big boi} expect not using a Laplace transform. Recall that equation \ref{big boi} in the frequency domain is given by
        \[ 
            U_t + D \w^2 U = \delta(t - \t) e^{i\w\x}.
        \]
        Note that $\delta(t - \t) = 0$ for $t \neq \t$ by definition and thus we can break up the ODEs into
        \[ 
            \begin{cases}
                U_t + D \w^2 U = 0, \quad t < \t,\\
                U_t + D \w^2 U = 0, \quad t > \t,
            \end{cases}
        \]
        which have the general solutions
        \[ 
            U(\w,t) = \begin{cases}
                C(\w) e^{-D\w^2 t}, \quad t < \t,\\
                D(\w) e^{-D\w^2 t}, \quad t > \t,\\
            \end{cases}
        \]
        subject to the initial condition $u(x,0)=0$ which gives $U(\w,0) = 0$. Since we assumed that $\t > 0$, then the initial condition only applies to the case when $t < \t$ which gives that $C(\w) = 0$ and thus 
        \[ 
            U(\w,t) = \begin{cases}
                0, \quad t < \t,\\
                D(\w) e^{-D\w^2 t}, \quad t > \t.\\
            \end{cases}
        \]
        To solve for $D(\w)$, we will find a matching condition across $t = \t$ by integrating across $t = \t$ by taking $\t^+$ and $\t^-$ to be on either side of $t = \t$. Observe that
        \[ 
            \int_{\t^-}^{\t^+} U_t + D \w^2 U  \d t = \int_{\t^-}^{\t^+} \delta(t - \t) e^{i\w\x} \d t = e^{i\w\x},
        \] 
        and
        \begin{align*}
            \int_{\t^-}^{\t^+} U_t + D \w^2 U  \d t &= \int_{\t^-}^{\t^+} U_t \d t + D \w^2 \cancel{\int_{\t^-}^{\t^+} U  \d t}\\
            &= U(\w,\t^+) - \cancel{U(\w,\t^-)}\\
            &= U(\w,\t^+) = D(\w)e^{-D\w^2 \t},
        \end{align*}
        were the first cancellation was due to $U$ being finite and the second since $U(\w,t) = 0$ when $t < \t$.
        Thus we have that
        \[ 
            D(\w)e^{-D\w^2 \t} = e^{i\w\x} \implies D(\w) = e^{i\w\x + D\w^2\t},
        \] 
        which gives that
        \[ 
            U(\w,t) = \begin{cases}
                0, \quad t < \t\\
                e^{i\w\x - D\w^2(t - \t)}
            \end{cases} = H(t - \t)e^{i\w\x - D\w^2(t - \t)},
        \]
        where $H$ denotes the Heaviside function. This is the same solution in frequency space we found prior and thus applying the inverse Fourier transform yields
        \[ 
            u(x,t) = \F^{-1}[u(\w,t)] = \frac{H(t - \tau)e^{-\paren{\frac{x-\x}{2D(t- \t)}}^2}}{\sqrt{4\pi D(t - \t)}}.
        \]

    \end{enumerate}

\end{solution}

%----------------------------------------------------------------------------------------------------%
%\vskip 20pt
\newpage


\end{document}