\documentclass[12pt]{report}

\usepackage{commands}
\usepackage{cancel}

\begin{document}

\large

\begin{center}
 Math 569 Homework 2\\
 Due April 26\\
 By Marvyn Bailly\\
\end{center}

\normalsize

\hrule

%---------------%
%---Problem 1---%
%---------------%

%--status--$

\begin{problem}
    Consider the wave equation
    \[ 
        c^2 u_{xx} - u_{tt} = 0,
    \]
    which is a special case of the general quasi-linear equation:
    \[ 
        a u_{xx} + 2bu_{xt} + cu_{tt} = f.
    \]
    Find the slope of each of the two characteristics:
    \[ 
        \dd{t}{x} = -z_1 ~\text{along $\alpha = $ constant characteristic}
    \]
    and
    \[ 
        \dd{t}{x} = -z_2 ~\text{along $\beta = $ constant characteristic.}
    \]

    Find the expression in terms of $x$ and $t$ for $\alpha$ and $\beta$ so that the wave equation simplifies to 
    \[ 
        u_{\alpha \beta} = 0.
    \]
    
\end{problem}

\begin{solution}

    \noindent
    Consider the wave equation
    \[ 
        c^2 u_{xx} = u_{tt},
    \]
    which is a special case of the general quasi-linear equation:
    \[ 
        au_{xx} + 2bu_{xt} + c'u_{tt} = f,
    \]
    where $a = c^2, b = 0, c' = - 1, \and f=0$. To classify the PDE, we will use the transformation on the independent variables
    \[ 
        \begin{cases}
            \alpha = \phi(x,t)\\
            \beta = \psi(x,t),
        \end{cases}
    \]
    which we require to be locally invertible. This implies that
    \[ 
        J(x,t) = \det \begin{bmatrix}
            \pp{\alpha}{x} & \pp{\alpha}{t}\\
            \pp{\beta}{x} & \pp{\beta}{t}\\
        \end{bmatrix} \neq 0.
    \]
    Using the chain rule, we can compute the first and second derivates to be
    \begin{align*}
        u_x &= u_\alpha \alpha_x + u_\beta \beta_x\\
        u_{xx} &= u_\alpha \alpha_{xx} + u_\beta \beta_{xx} + u_{\alpha \alpha} \alpha_x^2 + 2u_{\alpha \beta}\alpha_x\beta_x + u_{\beta \beta}\beta_x^2,\\
    \end{align*}
    and
    \begin{align*}
        u_t &= u_\alpha \alpha_t + u_\beta \beta_t\\
        u_{tt} &= u_\alpha \alpha_{tt} + u_\beta \beta_{tt} + u_{\alpha \alpha} \alpha_t^2 + 2u_{\alpha \beta}\alpha_t\beta_t + u_{\beta \beta}\beta_t^2.
    \end{align*}
    We now plug the transformation back into the wave equation to get
    \[ 
        A(\alpha,\beta)u_{\alpha \alpha} + 2B(\alpha, \beta)u_{\alpha \beta}+ C(\alpha,\beta)u_{\beta \beta} = F,
    \]
    where $F$ contains all the other contributions, not explicitly written and where
    \[ 
        \begin{cases}
            A = a \alpha_x^2 + 2b\alpha_x\alpha_t + c'\alpha_t^2 &= c^2 \alpha_x^2 - \alpha_t^2\\
            B = a \alpha_x \beta_x + b(\alpha_x \beta_t + \alpha_t \beta_x) + c' \alpha_t \beta_t &= c^2 \alpha_x \beta_x - \alpha_t \beta_t\\
            C = a \beta_x^2 + 2b\beta_x \beta_t + c'\beta_t^2 &= c^2 \beta_x^2 - \beta_t^2. 
        \end{cases}
    \]
    Now if we pick $\alpha$ and $\beta$ to satisfy the equation
    \[ 
        \alpha \varphi_x^2 + 2b \varphi_x \varphi_t + c' \varphi_t^2 = 0, 
    \]
    then we achieve that $A = C = 0$ which reduces the PDE to 
    \[ 
        2Bu_{\alpha \beta} = F.
    \]
    To determine $\alpha \and \beta$, we divide by $\varphi_t^2$ to obtain
    \[ 
        \alpha \paren{\frac{\varphi_x}{\varphi_t}}^2 + 2b \paren{\frac{\varphi_x}{\varphi_t}} + c' \varphi_t^2 = 0, 
    \]
    and setting $z = \frac{\varphi_x}{\varphi_t}$ we obtain the quadratic equation
    \[ 
        az^2 + 2bz + c' =  c^2z^2 - 1 = 0,
    \]  
    and since $b^2 - ac'= c^2 > 0$, the roots will be real and not equal. Furthermore, we can compute the roots to be 
    \[ 
        c^2z^2 - 1 = 0 \implies z_{1,2} = \pm \frac{1}{c}.
    \] 
    Since $z_1 \neq z_2$ and $z_1, z_2 \in \R$, we know that
    \[ 
        \begin{cases}
            \alpha_x = z_1 \alpha_y\\
            \alpha_y = z_2 \beta_y,
        \end{cases}
    \]
    and thus we see that $\alpha(x,y) =$ a constant and is one family of characteristics and $\beta(x,y)=$ constant which is a second family of characteristics. The slope of the characteristics in the $x-y$ plane are
    
    \[ 
        \dd{t}{x} = -z_1 = - \frac{1}{c},
    \]
    and
    \[ 
        \dd{t}{x} = -z_2 = \frac{1}{c}.
    \]
    Thus for $\alpha$ we have 
    \[ 
        \dd{\alpha}{x} = 0, \and \dd{t}{x} = -\frac{1}{c},
    \]
    which we can solve using the method of characteristics to find
    \[ 
        \alpha = x + ct,
    \]
    and similarly for $\beta$ we find 
    \[ 
        \beta = x - ct.
    \] 
    Next let's re-compute the chain rule with $\alpha$ and $\beta$ to get
    \begin{align*}
        u_x &= u_\alpha + u_\beta\\
        u_{xx} &= u_{\alpha \alpha} + 2u_{\alpha \beta} + u_{\beta \beta},\\
    \end{align*}
    and
    \begin{align*}
        u_t &= cu_\alpha - cu_\beta\\
        u_{tt} &= c^2u_{\alpha \alpha} - 2c^2u_{\alpha \beta} + c^2u_{\beta \beta}.
    \end{align*}
    Plugging this into the wave equation yields
    \begin{align*}
        0 &= c^2u_{xx} - u_{tt}\\
        &= c^2(u_{\alpha \alpha} + 2u_{\alpha \beta}\alpha_x\beta_x + u_{\beta \beta}) - ( c^2u_{\alpha \alpha} - 2c^2u_{\alpha \beta} + c^2u_{\beta \beta})\\
        &= (c^2 - c^2)u_{\alpha \alpha}+ (2c^2 + 2c^2)u_{\alpha \beta} +(c^2 - c^2)u_{\beta \beta}\\
        &= 4c^2 u_{\alpha \beta}\\
        &= 2 B u_{\alpha \beta},
    \end{align*} 
    where $B = 2c^2$ and $A = C = 0$. Thus we have found a transform that reduces the PDE to
    \[
        2Bu_{\alpha \beta} = 0 \iff u_{\alpha \beta} = 0, 
    \]
    since $B \neq 0$. Sorry for the roundabout way that I took to get here. 
\end{solution}

%----------------------------------------------------------------------------------------------------%
%\vskip 20pt
\newpage

%---------------%
%---Problem 2---%
%---------------%

%--status--$

\begin{problem}
    Use the Fourier transform method to solve the 2-D Laplace equation in the upper plane for the bounded solution:
    \[
        \nabla^2 u = 0, ~\text{in}~ y>0, -\infty < x <\infty
    \]
    which is subject to $u(x,0) = f(x)$ where $f(x)$ is of compact support; $u(x,y) \to 0$ as $x \to \pm \infty$.
\end{problem}

\begin{solution}

    \noindent
    Consider the 2-D Laplace equation in the upper plane
    \[ 
        \nabla^2 u = u_{xx} + u_{yy} = 0,
    \]
    for $y > 0$ and $-\infty < x < \infty$, subject to the initial condition $u(x,0) = f(x)$ where $f(x)$ is of compact support; $u(x,y) \to 0$ as $x \to \pm \infty$. We wish to apply the Fourier transform to solve this PDE. We begin by transforming the PDE and initial condition into the frequency domain. We first define
    \[ 
        U(\w,y) = \F[u(x,y)] = \int_{-\infty}^{\infty}u(x,y)e^{i\w x}dx.
    \]  
    Then using integration by parts twice we find that,
    \begin{align*}
        \F[u_{xx}] &= \int_{-\infty}^{\infty}u_{xx}e^{i\w x}\d x\\
        &= [u_x e^{i\w x}]_{-\infty}^{\infty} - i\w \int_{-\infty}^{\infty} u_x e^{i\w x}\d x\\
        &= \cancel{[u_x e^{i\w x}]_{-\infty}^{\infty}} - i\w \cancel{[ue^{i\w x}]_{-\infty}^\infty} - \w^2 \int_{-\infty}^{\infty} u e^{i \w x}\d x\\
        &= -\w^2 U,
    \end{align*}
    where the first is canceled as $u(x,y) \to 0$ as $x \to \pm \infty$ and the second term is canceled by making the assumption that $u_x \to 0$ as $|x| \to \infty$. We also compute
    \[ 
        \F[u_{yy}] = \int_{-\infty}^{\infty} u_{yy}e^{i\w x}\d x = \ppn{2}{}{y}\int_{-\infty}^{\infty}ue^{i\w x} \d x = U_{yy},
    \]
    and transform the initial condition
    \[ 
        U(\w, 0) = \F[u(x,0)] = \F[f(x)] = F(\w),
    \]
    Now we can reformulate the PDE in the frequency domain as the following ODE
        \[ 
            -\w^2 U + U_{yy} = 0.
        \]
    which has the general solution
    \[ 
        U(\w,y) = c_1(\w)e^{\w y} + c_2(\w)e^{-\w y}.
    \]
    Notice that if we also assume that $u(x,y) \to 0$ as $y \to \infty$, then we also require $U(\w,y) \to 0$ as $y \to \infty$. Thus we have two different cases: 1) if $\w > 0$, then $A(\w) = 0$ to satisfy the condition, or 2) if $\w < 0$, then $B(\w) = 0$ to satisfy the condition. Enforcing this condition we get the solution to be 
    \[
        U(\w,y) = \begin{cases}
            B(\w)e^{-\w y} &\w>0\\
            A(\w)e^{\w y} &\w<0,
        \end{cases}
    \]
    and enforcing the initial condition we find a particular solution to be
    \[
        U(\w,y) = \begin{cases}
            F(\w)e^{-\w y} &\w>0\\
            F(\w)e^{\w y} &\w<0.
        \end{cases}
    \]
    Now we wish to transform the solution out of the frequency space, so let's take the inverse Fourier transform which gives
    \begin{align*}
        u(x,y) &= \F^{-1}[U(\w,y)]\\
        &= \frac{1}{2\pi}\int_{-\infty}^{\infty}U(\w,y)e^{-i\w x}\d\w\\
        &= \frac{1}{2\pi}\paren{\int_{-\infty}^0 F(\w)e^{\w y}e^{-i \w x}\d\w + \int^{\infty}_0 F(\w)e^{-\w y}e^{-i \w x}\d\w}\\
        &= \frac{1}{2\pi} \int_{-\infty}^{\infty}F(\w)e^{-|\w|y}e^{-i\w x}\d\w\\
        &= \F^{-1}[F(\w)e^{-|\w|y}].
    \end{align*}
    Recall that the convolution theorem states
    \[ 
        \F^{-1}[gh] = \F^{-1}[g] * \F^{-1}[h],
    \]
    where $*$ denotes the convolution operation. Applying the theorem to our problem yields
    \[ 
        \F^{-1}[F(\w)e^{-|\w|y}] = \F^{-1}[F(\w )] * \F^{-1}[e^{-|\w|y}] = f(x) * \F^{-1}[e^{-|\w|y}].
    \]
    To find $\F^{-1}[e^{-|\w|y}]$, observe that
    \begin{align*}
        \F^{-1}[e^{-|\w|y}] &= \frac{1}{2\pi}\int_{-\infty}^{\infty} e^{-|\w|y}e^{-i\w x}\d\w\\
        &=\frac{1}{2\pi}\paren{\int_{-\infty}^{0}e^{\w(y-ix)}\d\w + \int_0^\infty e^{-\w(y+ix)}\d\w }\\
        &=\frac{1}{2\pi}\paren{\frac{1}{y-ix} + \frac{1}{y+ix}}\\
        &= \frac{1}{2\pi}\cdot \frac{2y}{x^2 + y^2}\\
        &= \frac{1}{\pi} \cdot \frac{y}{x^2 + y^2}. 
    \end{align*}
    Thus we have that
    \[
        u(x,y) = f(x) * \frac{1}{\pi} \cdot \frac{y}{x^2 + y^2} = \frac{1}{\pi}\int_{-\infty}^{\infty}f(t) \frac{y}{(x-t)^2 + y^2}\d t.
    \]
    Now let's double check the assumptions we made. As $y \to \infty$, 
    \[ 
        \frac{y}{(x-t)^2 + y^2} \to 0,
    \]
    and thus $u(x,y ) \to 0$ as $y \to \infty$ as the integrand tends to zero. We also have that
    \[ 
        u_x(x,y) = \frac{1}{\pi}\int_{-\infty}^{\infty}f(t) \frac{y(x - t)}{((x-t)^2 + y^2)^2}\d t,
    \] 
    and thus $u_x(x,y) \to 0$ as $|x| \to \infty$ as the integrand tends to zero.

    \end{solution}
    
%----------------------------------------------------------------------------------------------------%
%\vskip 20pt
\newpage

%---------------%
%---Problem 3---%
%---------------%

%--status--$

\begin{problem}
    Solve the following problem in two ways:
    \[ 
        \pp{}{t}u = \ppn{2}{}{x}u, ~~ t>0; ~0<x<\infty,
    \]
    subject to $u(x,0)=0$, $u(x,t)$ bounded as $x\to \infty$ and $u(0,t) = T_0$ a constant, $t>0$.
    \begin{enumerate}
        \item [(a)]
        By the method of similarity transformation. Look for the value of $\alpha$ such that the PDE reduces to an ode in $\eta$, $\eta = x/t^\alpha$.


        \item [(b)]
        By an integral transform in t, in this case a Laplace transform (You can use a table of Laplace transform to do the inverse transform).
    
    \end{enumerate}
\end{problem}

\begin{solution}

    \noindent
    Consider the problem
    \begin{align*}
        u_t &= u_{xx}, \quad t>0, 0 < x < \infty\\
        u(x,0) &= 0\\
        u(0,t) &= T_0, 
    \end{align*}
    where $u(x,t)$ is bounded as $x \to \infty$ and $T_0$ is a constant.

    \begin{enumerate}
        \item [(a)]
        We wish to solve the problem using the method of similarity transformation. Consider the similarity transformation
        \[
            \eta = \frac{x}{t^\alpha},
        \]    
        then we have the following chain rules 
        \[ 
            \pp{u}{t} = \pp{\eta}{t}\pp{u}{\eta} = \frac{-\alpha x}{t^{\alpha + 1}}\pp{u}{\eta} = - \frac{\alpha}{t}\eta \pp{u}{\eta},
        \]
        and
        \begin{align*}
            \pp{u}{x} &= \pp{\eta}{x}\pp{u}{\eta} = \frac{1}{t^\alpha}\pp{u}{\eta},\\
            \ppn{2}{u}{x} &= \pp{}{x}\paren{\frac{1}{t^\alpha}\pp{u}{\eta}} = \frac{1}{t^\alpha} \paren{\pp{\eta}{x}\pp{}{\eta}\paren{\pp{u}{\eta}}} = \frac{1}{t^{2\alpha}}\ppn{2}{u}{\eta}.
        \end{align*} 
        Plugging these into our equation 
        \[
            - \frac{\alpha}{t}\eta \pp{u}{\eta} = \frac{1}{t^{2\alpha}}\ppn{2}{u}{\eta},
        \]
        noticing that if we let $\alpha = 1/2$, the $t$ variable is eliminated. Thus 
        \[
            \eta = \frac{x}{\sqrt{t}},
        \]
        and our problem has become an ODE of the form 
        \begin{align*}
            -\frac{\eta}{2} \pp{u}{\eta} = \ppn{2}{u}{\eta},\\
            u(0) = T_0,
        \end{align*} 
        and $u(\eta) \to 0$ as $\eta \to \infty$. To solve the ODE, let $v = u'$ which yields
        \[ 
            \frac{v'}{v} = \frac{-1}{2},
        \] 
        and solving using separation of variables gives the solution to be
        \[
            v(\eta) = c_1 e^{-\eta^2/4}.
        \]
        Thus we have found
        \[ 
            u(\eta) = \int_0^\eta c_1 e^{-\eta^2/4} \d \eta = c_1 \text{erf}\paren{\frac{\eta}{2}} + c_2.
        \]
        Now we can enforce that $u(0) = T_0$ and $u(\eta) \to 0$ as $\eta \to \infty$. Recalling that $\text{erf}(0) = 0$ and $\text{erf}(\eta) \to 1$ as $\eta \to \infty$. The first condition gives $c_2 = T_0$ and the second condition gives $c_1 = -c_2 = -T_0$. Thus 
        \[
            u(\eta) =-T_0 \text{erf}\paren{\frac{\eta}{2}} + T_0 = \text{erfc}\paren{\frac{\eta}{2}},
        \]
        and undoing our transform gives the solution to be 
        \[
            u(x,t) = T_0 \text{erfc}\paren{\frac{x}{2\sqrt{t}}}.
        \]


        
        \item [(b)]
        Next we wish to solve the problem using a Laplace transform. Let $\tilde{u}(x,s)$ be the Laplace transform of the solution
        \[
            \tilde{u}(x,s) = \L[u(x,t)] = \int_0^\infty e^{-st}u(x,t)\d t.
        \]
        Then we have
        \[
            \L[u_{xx}] = \int_0^\infty e^{-st} \ppn{2}{}{x}u(x,t)\d t = \ppn{2}{}{x} \int_0^\infty e^{-st} u(x,t)\d t = \tilde{u}_{xx}(x,s),
        \]
        and using integration by parts
        \begin{align*}
            \L[u_{t}] &= \int_0^\infty \pp{}{t} u e^{-st}\d t\\
            &= \cancel{[u e^{-st}]_0^\infty} + s \int_0^\infty u e^{-st}\d t\\
            &= s \tilde{u}(x,s),
        \end{align*}
        if we make the assumption that $ue^{-st} \to 0$ as $t \to \infty$. Plugging these into our PDE we have
        \[ 
            s \tilde{u} = \tilde{u}_{xx},
        \]
        which is an ODE where $s>0$. We now Laplace transform the initial condition
        \[
            \tilde{u}(0,s) = \int_0^\infty u(0,t)e^{-st}\d t = \int_0^\infty T_0 e^{-st} \d t = \frac{T_0}{s}.
        \]
        Now the general solution to the ODE is
        \[
            \tilde{u}(x,s) = c_1(s) e^{\sqrt{s}x} + c_2(s) e^{-\sqrt{s}x}.
        \]
        To find a particular solution, we make the assumption that $u(x,t) \to 0$ as $x \to \infty$ which implies that $c_1(s) = 0$. Since $\tilde{u}(0,s) = \frac{T_0}{s}$, we have that $c_2 = \frac{T_0}{s}$. Thus the solution becomes
        \[
            \tilde{u}(x,s) = \frac{T_0}{s}e^{-\sqrt{s}x}.
        \]
        Using a table of Laplace transforms (Mathematica) we find the inverse transform of the solution 
        \[
            u(x,t) = \L^{-1}[\tilde{u}(x,s)] = T_0\text{erfc}\paren{\frac{x}{2\sqrt{t}}},
        \]
        which is the solution we found in part (a). Note that since $\frac{x}{2\sqrt{t}} \to 0$ as $t \to \infty$ and the $\text{erfc}(0)=1$, we can verify that
        \[
            \lim_{t \to \infty} ue^{-st} = \lim_{t \to \infty}T_0e^{-st} = 0,
        \] 
        since $s>0$. We also have that $u(x,t) \to 0$ as $x \to \infty$ since $\text{erfc}{x} \to 0$ as $x \to \infty.$



    \end{enumerate}
\end{solution}

%----------------------------------------------------------------------------------------------------%
%\vskip 20pt
\newpage

\end{document}