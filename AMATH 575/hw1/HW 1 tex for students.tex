\documentclass{article}
\usepackage[utf8]{inputenc}

\newcommand{\bei}{\begin{itemize}}
\newcommand{\ii}{\begin{itemize} \item}

\newcommand{\eei}{\end{itemize}}

\newcommand{\ben}{\begin{enumerate}}
\newcommand{\een}{\end{enumerate}}

\newcommand{\beq}{\begin{equation}}
\newcommand{\eeq}{\end{equation}}

\begin{document}

\begin{center}
AMATH 575 \\
Problem set 1\\[.3in]
\end{center}
\noindent {\bf Working together is  welcomed.  Please do not refer to previous years' solutions.} 


\newcounter{bean}

\begin{list}
   {\Roman{bean}}
    {\usecounter{bean}\setlength{\rightmargin}{\leftmargin}}

\item Consider the system in the phase plane
\begin{eqnarray*} 
	\dot x &=& f(x) \\
	\dot y &=& g(y) 
\end{eqnarray*} 
where $f(x)$ is a continuously differentiable real-valued function of $x$ alone and $g(y)$ is a continuously differentiable real-valued function of $y$ alone (and $x$ and $y$ are both 1-dimensional coordinates defining a plane).  Define an oscillatory solution as a trajectory $(x(t), y(t))$ such that $x(t)$ and $y(t)$ are not constant it time and, for any integer $N$, $x(t+N T) = x(t)$ and $y(t+N T) = y(t)$.  Here, $T$ is the period of the oscillation.

\smallskip 
(a) Answer YES or NO and give a simple proof or example:  Can a system of this form produce an oscillatory solution?  (b) Then, repeat this question, but for the discrete time map 
\begin{eqnarray*} 	x_{n+1} &=& f(x_n) \\
	\dot y_{n+1} &=& g(y_n) 
\end{eqnarray*}

\item Consider the 2-D systems below.  Find all equilibria and determine where they are lyapunov stable, asymptotically stable, or neither.  Here $\mu$ is an arbitrary real parameter, so make sure to give answers valid for each relevant range of $\mu$:
    \bei
    \item $\dot x = 0$, $\dot y = \mu x$ 
    \item $\dot x = 0$, $\dot y = \mu y$ 
    \eei


\item Consider the 2-D system:
\begin{eqnarray*} 
	\dot x &=& -y + \mu (x^2 + y^2)x  \\
	\dot y &=& x + \mu (x^2 + y^2) y
\end{eqnarray*}
where $\mu$ is an arbitrary real parameter.  Hint:  transform to polar coordinates and obtain an exact solution.
\item[a]  For all possible values of $\mu$, find all fixed points, and determine whether they are lyapunov stable, asymptotically stable, or neither.

\item[b]  For all possible values of $\mu$, and all possible initial values, determine the maximum duration in both forward and inverse time for which a solution exists.


 \item The van der Pol oscillator.  Consider:  $$\frac{d^2x}{dt^2} + (x^2 - v) \frac{dx}{dt} + x =0 $$ where $v$ is a parameter that can take any real value.
\item[a]  Find all fixed points, and the Jacobian evaluated as these fixed point(s).
\item[b]  State whether the fixed point(s) are lyapunov stable, asymptotically stable, or neither for all possible values of $v$.


\item A general description of a network of $N$ nonlinearly coupled units is given by
\beq
\frac {d u_i}{dt} = -u_i + \sum_{j=1}^N w_{ij} g(u_j)
\eeq
Here, $u_i$ is the activity of the $i^{th}$ unit, and the matrix $w$ gives the connection weights among these units; in particular, $w_{ij}$ is the connection weight between unit $j$ and unit $i$.  Finally, $g(\cdot)$ is a monotonically increasing function that describes how the strength of interaction between units depends on their activities.

\bei
\item For LINEAR interactions:  $g(y) = y$, write down a simple bound on the entries $w_{ij}$, based on the Gershgorin circle theorem, that guarantees that the origin will be a stable equilibrium.


\item  For NONLINEAR interactions $g(\cdot)$, consider the ``energy function" \beq H=-1/2 \sum_{ij} w_{ij} V_i V_j + \sum_i \int_0^{V_i} g^{-1}(V) dV \eeq
where $V_i=g(u_i)$.  [a] If the matrix $w$ is symmetric, show the following bound on the time evolution of the energy:  $$\frac{dH}{dt} \le 0 \; \;.$$  Your answer should be valid for a smooth monotonically increasing function $g(\cdot)$.  [b] Then, let $\bar u$ be an equilibrium for the system.  What additional requirements on $H$ would imply that $\bar u$ is asympotically stable?  [c]  Take $g(x) = \tanh(x)$. Construct a simple example of $w$ for which you can find an equilibrium $\bar u$ and demonstrate that your function $H$ implies that it is asymptotically stable.  A numerical approach is suggested, and rigorous arguments are not needed, though of course if you wish to use analysis instead that is just fine.  [d] Finally, does the bound  $$\frac{dH}{dt} \le 0 \; \;$$ also hold in general for anti-symmetric $w$? 
\eei

\item A specific case of the Lorenz equations is given by
\begin{equation}
\left\{\begin{array}{l}
x^{\prime}=10(-x+y) \\
y^{\prime}=r x-y-x z \\
z^{\prime}=-\frac{8}{3} z+x y
\end{array}\right.
\end{equation}
\item [a] For varying r, find all equilibrium points and discuss
their stability.
\item [b] Calculate up to second order terms the local invariant manifolds $W^u$, $W^s$ and $W^c$
for the fixed point at the origin of the Lorenz equations when $r=1$.



\end{list}








\end{document}





