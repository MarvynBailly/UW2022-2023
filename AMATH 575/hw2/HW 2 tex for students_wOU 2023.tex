\documentclass{article}
\usepackage[utf8]{inputenc}

\newcommand{\bei}{\begin{itemize}}
\newcommand{\ii}{\begin{itemize} \item}

\newcommand{\eei}{\end{itemize}}

\newcommand{\ben}{\begin{enumerate}}
\newcommand{\een}{\end{enumerate}}

\newcommand{\beq}{\begin{equation}}
\newcommand{\eeq}{\end{equation}}

\begin{document}


\begin{center}
AMATH 575 \\
Problem set 2\\[.3in]
\end{center}

\noindent {\bf Working together is  welcomed.  Please do not refer to previous years' solutions.} 


\begin{enumerate}
   
 \item Project preparation!  [a]  Provide the citation for the paper you're planning to base your presentation on, and a list of your planned group members.  [b]  Write two sentences describing the main techniques from dynamical systems, broadly interpreted, used in the paper.  [c]  Write two sentences describing the conclusions that the paper found.  [d]  Write one sentence with an idea for a possible extension to the paper that you might explore.  
   
\item Consider the dipole system again

\begin{equation}
\left\{\begin{array}{l}
\dot x= x^2 - y^2  \\
\dot y = 2 x y 
\end{array}\right.
\end{equation}

[a] What is the center manifold (and what is its dimension)?  [b] Find a one-dimensional invariant manifold. 


\item Consider the system 
\begin{equation}
\left\{\begin{array}{l}
\dot x= -x \\
\dot y = y + \alpha x^3 
\end{array}\right.
\end{equation}
for $\alpha>0$.  Find an exact expression for the global stable and unstable manifolds (i.e., not in terms of a polynomial expansion).

\item Consider linear map on the torus $(x,y) \in \mathbb{T}^2 =  \mathbb{R}^2  /  \mathbb{Z}^2$
\begin{equation}
\left\{\begin{array}{l}
x_{n+1} = x_n + y_n \\
y_{n+1} = x_n + 2 y_n  
\end{array}\right.
\end{equation}
Find the global unstable and stable manifolds of the fixed point at the origin.  How many times do they intersect?

\item Consider the system 
\begin{equation}
\left\{\begin{array}{l}
\dot x= \alpha x^2 -xy \\
\dot y = -y + x^2 
\end{array}\right.
\end{equation}
for all possible real values of  $\alpha$.  Use a polynomial expansion to compute the center manifold of the fixed point at the origin.  Write down the approximate flow on the center manifold.  Draw conclusions about the stability of the origin.  Please make sure your answers cover all possible real values of  $\alpha$.
\item
Wiggins 5.3.4.

\item{Homoclinic tangle} Consider the standard map
\begin{equation}
    \left\{\begin{array}{l}
\theta_{n+1}=\theta_{n}+I_{n}-\frac{K}{2 \pi} \sin 2 \pi \theta_{n} \bmod 1 \\
I_{n+1}=I_{n}-\frac{K}{2 \pi} \sin 2 \pi \theta_{n}
\end{array}\right.
\end{equation}
In what follows, we will examine the Standard Map with K = 1. Choose your
favorite programming language (Python, MATLAB, Maple, etc) to iterate forward
and backward along the manifolds, following the steps below. You will find a homoclinic tangle. You are welcome to start with the ipython notebook provided in class.\\

[a]  Check that this map has a fixed point at $(\theta, I) = (1/2, 1)$. \\

[b] Determine the linear stability of this fixed point, including its linear stable
and unstable eigenspaces.\\

[c] Take many points near the fixed point on the linear unstable eigenspace and
propagate them forward a number of iterations, until you have a well-resolved
view of the global structure of the global unstable manifold. \\

[d] Construct the inverse map.\\

[e] Take many points near the fixed point on the linear stable eigenspace of the
map and propagate them forward using the inverse map a number of iterations,
until you have a well-resolved view of the global structure of the global stable
manifold.\\ 

[f] Continue for sufficiently many iterations to view the homoclinic tangle.
Plot both
stable and unstable manifolds, in different colors.\\

[g] Choose a single point inside this homoclinic tangle, close to the fixed point.
Plot the orbit of this point (how the point evolves as you iterate), in a third
color, combined with the homoclinic tangle. Discuss this evolution, and how it is shaped by the structure of the homoclinic tangle.  

Make sure to hand in all plots and code, together with any hand-calculations for this problem.

\item \textbf{Periodic orbit in a model for oscillations in glycolysis} Yeast cells break down
sugar by glycolysis, and there is a simple model for this:
\begin{equation}
\left\{\begin{array}{l}
\dot{x}=-x+a y+x^{2} y \equiv f(x, y) \\
\dot{y}=b-a y-x^{2} y \equiv g(x, y)
\end{array}\right.
\end{equation}
where x represents ADP adenosine diphosphate and y is F6P fructose-6-phosphate
concentrations $(x , y \ge 0)$. Assume $0 < a \le 1/8$. By constructing a trapping region
and using Poincare-Bendixson theorem, find the range of b where one can guarantee
the existence of a stable periodic orbit.  Hint:  you may wish to plot the vectorfield using the ipython notebook form class, or pplane (matlab) or other software, to get inspiration for where you would like to define the trapping region.

\item \textbf{Predator-prey system} Consider the system
\begin{equation}
\left\{\begin{array}{l}
\dot{x}=x(a-by) \\
\dot{y}=y(-c+dx)
\end{array}\right.
\end{equation}
where the parameters $a,b,c,c$ are all assumed to be positive. Additionally, since we are dealing with populations, we only consider $x,y \geq 0$.\\

[a] Find the fixed points of this system, and determine whether they are lyapunov stable, asymptotically stable, or inconclusive based on the eigenvalues (i.e. the linearized system).\\

[b] Draw the $x$-nullclines and the $y$-nullclines of the system. They should separate the region $x,y>0$ into four basic regions. Add a rough sketch of the vector fields in each of these regions to your drawing. What can you infer about the solutions of this system? \\

[c] Next, let us search for a Lyapunov function $L$ for the system. Using the trick of {\it separation of variables}, we will look for a function of the form
\begin{equation}
    L(x,y) = F(x) + G(y).
\end{equation}
Additionally, we will use the facts that $dL/dt := 0$ along solutions of the system, and that $x$ and $y$ are independent variables to obtain solvable ODEs for $F(x)$ and $G(y)$. \\

[d] Finally, use the Lyapunov function to answer any remaining questions about the stability of the fixed points of our original system. (Note that strictly speaking the function $L(x,y)$ that you obtain might need to be shifted by a constant value to make it a true a Lyapunov function.) 


\end{enumerate}


\end{document}