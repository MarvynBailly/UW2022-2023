\documentclass{article}
\usepackage[utf8]{inputenc}

\newcommand{\bei}{\begin{itemize}}
\newcommand{\ii}{\begin{itemize} \item}

\newcommand{\eei}{\end{itemize}}

\newcommand{\ben}{\begin{enumerate}}
\newcommand{\een}{\end{enumerate}}

\newcommand{\beq}{\begin{equation}}
\newcommand{\eeq}{\end{equation}}

\begin{document}


\begin{center}
AMATH 575 \\
Problem set 4\\[.3in]
\end{center}

\noindent {\bf Working together is  welcomed.  Please do not refer to previous years' solutions.} 

\newcounter{bean}

\begin{list}
   {\Roman{bean}}
    {\usecounter{bean}\setlength{\rightmargin}{\leftmargin}}

 \item Please be working on your next project presentation, scheduled for May 24 and 26.   The final paper will then be due on June 9.   
 
 \item  Consider the ``all-to-all'' coupled system of pulse-coupled phase oscillators on the N-dimensional torus, with coupling strength $\epsilon>0$, from class:
 \begin{equation}
\dot \theta_i = \omega + \epsilon z(\theta_i) \frac{1}{N} \sum_{j=1}^N g(\theta_j) \, \, \,  mod \; \; 2 \pi  
\end{equation}
$i=1...N$.  Let $z(\theta_i)=A \sin \theta + B \cos \theta $, which we noted in class corresponds to Hopf, generalized Hopf, and to saddle-node on a periodic orbit bifurcations, which are the most common co-dimension 1 bifurcations to periodic orbits.  Let $g(\theta) = \sum_{k=1}^\infty a_k \sin(k \theta) + b_k \cos(k \theta)$, a totally general ``impulse" function describing the coupling from oscillator j.  Beginning with the same coordinate transformation as in class, compute the averaged system
 \begin{equation}
\dot \psi_i =  \epsilon  \frac{1}{N} \sum_{j=1}^N f(\psi_j - \psi_i) \, \, \,  mod \; \; 2 \pi  
\end{equation} 
Recall that the conclusions of the averaging theorem on how the latter equation approximates the first hold here, making the latter equation a useful approximation.
[a] Find a general explicit expression for $f$, involving the constants $A,B, a_k, b_k$ above for appropriate $k$.
[b] Building from a previous homework, find a general condition on these constants that guarantees that the averaged system will be a gradient dynamical system.
[c] Find a general condition on these constants that guarantees that any solution $\psi_i \equiv k$ $\forall k$ is a fixed point for the averaged system.  These are referred to as synchronized solutions. Compute the Jacobian for these fixed points , and write down the dimension of the stable, unstable, and center manifolds for all possible choices of the  constants $A,B, a_k, b_k$ .

\item Compute the normal form, up to order two, for a two-dimensional flow with linear part (Jacobian, in real Jordan form) $$
   J=
  \left[ {\begin{array}{cc}
   1 & 0 \\
   0 & \lambda \\
  \end{array} } \right] $$ where $\lambda \neq 0$ is an arbitrary real parameter.    Note that the case you will study, $\lambda \neq 0$, is for a hyperbolic fixed point that is a saddle or a source.  Make sure to cover all of the possible cases; the normal form may differ for different values of $\lambda$.  As a comment, your result will relate nicely to Sternberg's Theorem (not covered in class):  if $\lambda_j$ are eigenvalues of $J$, the full flow can be linearized by a diffeomorphism if $\sum_{j=1}^n m_j \lambda_j \neq 0$ for all integers $m_j$.


\item Determine the Takens-Bogdanov normal form to third order.

\item In homework 1 we have the Lorenz equations
\begin{equation}
\left\{\begin{array}{l}
x^{\prime}=10(-x+y) \\
y^{\prime}=r x-y-x z \\
z^{\prime}=-\frac{8}{3} z+x y
\end{array}\right.
\end{equation}
Characterize the bifurcation when $r=1$.

\item Consider the one-parameter family of one-dimensional maps,
\begin{equation}
    x \mapsto x^2 + c,
\end{equation}
where $c$ is a real-valued parameter.
    \begin{enumerate}
        \item Find the fixed points of this system. For which values of $c$ do they exist? Determine the stability of these fixed points and their dependency on the value of $c$. Determine if there is a bifurcation, and find the bifurcation point.
        \item Focusing on the value $c = -3/4,$ compute $f'_{-3/4}(p_{-})$, where $f_c(x) = x^2 + c$ and $p_{-}$ is the smaller of the two fixed points at this value of $c.$ Convince yourself that as $c$ descends through $-3/4$, we see the emergence of an (attracting) 2-cycle. This is a period doubling bifurcation!
        \item Solve for the period two points by considering the fixed points of the function $f_c^2(.) = f_c(f_c(.)),$ and the domain of $c$ for which the original system has a fixed point. 
    \end{enumerate}

 \end{list} 


\end{document}