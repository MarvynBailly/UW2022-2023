\documentclass[12pt]{report}

\usepackage{amssymb, fullpage, amsmath}
\usepackage{graphicx}

\newtheorem{problem}{Problem}

\newenvironment{solution}[1][\it{Solution}]{\textbf{#1. } }{$\square$}

\graphicspath{ {./} }

\pagestyle{empty}

\def\Z{{\mathbb Z}}
\def\Q{{\mathbb Q}}
\def\C{{\mathbb C}}
\def\R{{\mathbb R}}
\def\N{{\mathbb N}}

\begin{document}

\large

\begin{center}
 Math 567 Homework 1\\
 Due October 12 2022\\
 By Marvyn Bailly\\
\end{center}

\normalsize

\hrule



%---------------%
%---Problem 1---%
%---------------%

%--Done--$

\begin{problem}
    Express each of the following in polar exponential form:
    \begin{enumerate}
        \item [b.] $-i$
        \item [c.] $1 + i$
        \item [d.] $\frac{1}{2} + \frac{\sqrt{3}}{2}i$
    \end{enumerate}
\end{problem}

\begin{solution}

    \noindent
    \begin{enumerate}
        \item [b.] Let $r = 1$ and $\theta = \frac{3 \pi}{2}$. Then $z = -i = e^{\frac{3\pi i}{2} + 2\pi i k}$ for $k=0,\pm 1, \pm 2, \dots$. 
        \item [c.] Let $r = \sqrt{2}$ and $\theta = \frac{\pi}{4}$. Then $z = \sqrt{2}e^{\frac{\pi}{4}i + 2k\pi i}$ for $k=0,\pm 1, \pm 2, \dots$.
        \item [d.] Let $r = \sqrt{\frac{1}{4} + \frac{3}{4}} = \sqrt{1} = 1$ and $\theta = \frac{\pi}{3}$. Then $z=e^{\frac{\pi}{3}i + 2k\pi i}$ for $k=0,\pm 1, \pm 2, \dots$. 
    \end{enumerate}
\end{solution}

%----------------------------------------------------------------------------------------------------%
%\vskip 20pt
%\newpage

%---------------%
%---Problem 2---%
%---------------%

%----$

\begin{problem}
    Express each of the following in the form of $a+bi$, where $a$ and $b$ are real.
    \begin{enumerate}
        \item [a.] $e^{2+i\pi/2}$
        \item [b.] $\frac{1}{1+i}$
        \item [c.] $(1+i)^3$
        \item [d.] $|3+4i|$
        \item [e.] $\cos(i \pi / 4 + c)$ for some real $c$ 
    \end{enumerate}
\end{problem}

\begin{solution}

    \noindent
    \begin{enumerate}
        \item [a.] $e^{2+i\pi/2} = e^{2}e^{i\pi/2}=e^{2}\left(\cos(\frac{\pi}{2}) + i \sin(\frac{\pi}{2})\right) = ie^{2}.$
        \item [b.] $\frac{1}{1+i} = \frac{1}{z} = \frac{\bar{z}}{|z|^2} = \frac{1-i}{2} = \frac{1}{2} - i \frac{1}{2}.$
        \item [c.] $(1+i)^3 = (1+i)(1+i)(1+i) = (1+2i-1)(1+i) = 2i(1+i) = 2(i - 1) = -2 + i2.$
        \item [d.] $|3+4i| = \sqrt{\Re(3+4i)^2 + \Im(3+4i)} = \sqrt{3^2 + 4^2} = \sqrt{25} = 5$
        \item [e.] Let c be real, then we can rewrite as following,
        \begin{align*}
            \cos(i \frac{\pi}{4} + c) &= \frac{e^{i(i \frac{\pi}{4} + c)}+e^{-i(i \frac{\pi}{4} + c)}}{2}\\ 
            &= \frac{1}{2}\left(e^{(-\frac{\pi}{4} + ic)}+e^{(\frac{\pi}{4} - ic)}\right)\\
            &= \frac{1}{2}\left(e^{(-\frac{\pi}{4})}e^{(ic)}+e^{(\frac{\pi}{4})} e^{(-ic)} \right)\\
            &= \frac{1}{2}\left(e^{(-\frac{\pi}{4})}(\cos(c)+i\sin(c))+e^{(\frac{\pi}{4})}(\cos(c)-i\sin(c))  \right)\\
            &= \frac{1}{2}\left(e^{(-\frac{\pi}{4})}\cos(c)+i\sin(c)e^{(-\frac{\pi}{4})}+e^{(\frac{\pi}{4})}\cos(c)-ie^{(\frac{\pi}{4})}\sin(c)  \right)\\
            &= \frac{1}{2}\cos(c)\left(e^{(-\frac{\pi}{4})}+e^{(\frac{\pi}{4})}\right)+i\frac{1}{2}\sin(c)\left(e^{(-\frac{\pi}{4})}-e^{(\frac{\pi}{4})}\right)\\
            &= \frac{1}{2}\left(\cos(c)\left(e^{(-\frac{\pi}{4})}+e^{(\frac{\pi}{4})}\right)+i\sin(c)\left(e^{(-\frac{\pi}{4})}-e^{(\frac{\pi}{4})}\right)\right)\\
            &=\cos(c)\cosh\left(\frac{\pi}{4}\right) - i\sin(c)\sinh\left(\frac{\pi}{4}\right)\\
        \end{align*}
    \end{enumerate}
\end{solution}

%----------------------------------------------------------------------------------------------------%
%\vskip 20pt
%\newpage
%---------------%
%---Problem 3---%
%---------------%

%--status--$

\begin{problem}
    Solve for the roots of the following equation: 
    \begin{enumerate}
        \item [a.] $z^3 = 4$
        \item [b.] $z^4 = -1$
    \end{enumerate}
\end{problem}

\begin{solution}
    \noindent
    \begin{enumerate}
        \item [a.] 
        Consider $z^3 = 4$. To find the roots we can use the roots of unity method which gives that the roots will be of form, $$z = \sqrt[3]{4}e^{2k\pi i/3}$$ for $n=1,2,3$. Thus the roots are $\sqrt[3]{4}e^{2\pi i/3},\sqrt[3]{4}e^{4\pi i/3},\sqrt[3]{4}$

        \item [b.] 
        Consider $z^4 = -1 \implies z^4 + 1 = 0$. To find the roots, we can rewrite the complex number in the form $z^4 = -1 = |-1|e^{i\pi}e^{i2\pi n}$ where $n \in \Z$. Then we get,
        $$z=e^{i\pi\frac{(2n+1)}{4}}$$
        which gives the roots to be,
        $$e^{i\frac{\pi}{4}},e^{i\frac{3\pi}{4}},e^{i\frac{5\pi}{4}},e^{i\frac{7\pi}{4}}$$
         
    \end{enumerate}
\end{solution}

%----------------------------------------------------------------------------------------------------%
%\vskip 20pt
%\newpage
%---------------%
%---Problem 4---%
%---------------%

%--status--$

\begin{problem}
    Establish the following result: 
    \begin{enumerate}
        \item [a.] $(z+w)^*=z^* + w^*$
        \item [d.] $\text{Re}(z) \leq |z|$
        \item [e.] $|wz^* + w^*z| \leq 2|wz|$
        \item [f.] $|z_1z_2| = |z_1||z_2|$
    \end{enumerate}
\end{problem}

\begin{solution}
    \noindent
    \begin{enumerate}
        \item [a.] 
        Let $z,w \in \C$ such that $z=a+ib$ and $w=c+id$.
        Then,
        \begin{align*}
            (z+w)^* &= ((a+ib) + (c+id))^*\\
                    &= ((a+c) + i(b+d))^*\\
                    &= (a+c) - i(b+d)\\
                    &= (a+c) +(-ib - id)\\
                    &= (a-ib) + (c - id)\\
                    &= z^* + w^*
        \end{align*}
        Thus we have shown that $(z+w)^* = z^* + w^*$.
        \item [d.]
        Let $z \in \C$ such that $z=a+ib$.
        Then we have that,
        $$\Re(z) = a \leq |a| = \sqrt{a^2} \leq \sqrt{a^2 + b^2} = |z|.$$
        \item [e.]
        Let $w,z \in \C$ such that $z=a+ib$ and $w=c+id$.
        Then,
        \begin{align*}
            |wz^* + w^*z| &= |(c+id)(a+ib)^* + (c+id)^*(a+ib)|\\
                          &= |(c+id)(a-ib) + (c-id)(a+ib)|\\
                          &= |ca - icb + iad + db + ca + icb -iad + db|\\
                          &= |2ca +2db|\\
                          &= 2\sqrt{(ca + db)^2}\\
                          &= 2\sqrt{(ca)^2 + 2cadb + (db)^2}
        \end{align*}
        \begin{align*}
            2|wz| &= 2 |(c+id)(a+ib)|\\
                  &= 2 |a c - b d + i (a d + b c) |\\
                  &= 2 \sqrt{(ac - bd)^2 + (ad + bc)^2}\\
                  &= 2 \sqrt{ a^2 c^2 - 2 a b c d + b^2 d^2 + a^2 d^2 + 2 a b c d + b^2 c^2}\\
                  &= 2 \sqrt{ (ac)^2 + (bd)^2 + (ad)^2 + (bc)^2}\
        \end{align*}
        Thus it is left to show that $2cabd \leq (bd)^2 + (ad)^2$.
        If we let $A = ad$ and $B = bd$ then we want to show that $2AB \leq A^2 + B^2$. But we know that $(A-B)^2 = A^2 - 2AB + B^2 \geq 0$
        and thus $2AB \leq A^2 + B^2$ and we have shown that $(z+w)^* = z^* + w^*$. 

        \item [f.] $|z_1z_2| = |z_1||z_2|$.
        
        Let $z_1,z_2 \in \C$ such that $z_1 = x_1 + iy_1$ and $z_2 = x_2 + i y_2$. Then,
        \begin{align*}
            |z_1z_2| &= |(x_1 + iy_1)(x_2 + iy_2)|\\
                    &= |i x_2 y_1 + i x_1 y_2 + x_1 x_2 - y_1 y_2|\\
                    &= |(x_1 x_2 - y_1 y_2) + i(x_2 y_1 + x_1 y_2)|\\
                    &= \sqrt{(x_1 x_2 - y_1 y_2)^2 + (x_2 y_1 + x_1 y_2)^2}\\
                    &= \sqrt{-2 x_1 x_2 y_1 y_2 + x_1^2 x_2^2 + y_1^2 y_2^2 + x_2^2 y_1^2 + 2 x_1 x_2 y_2 y_1 + x_1^2 y_2^2}\\
                    &= \sqrt{x_1^2 x_2^2 + y_1^2 y_2^2 + x_2^2 y_1^2 + x_1^2 y_2^2}\\
                    &= \sqrt{(x_1^2 + y_1^2)(x_2^2 + y_2^2)}\\
                    &=\sqrt{(x_1^2 + y_1^2)}\sqrt{(x_2^2 + y_2^2)}\\
                    &=|z_1||z_2|
        \end{align*}
    \end{enumerate}
\end{solution}

%----------------------------------------------------------------------------------------------------%
%\vskip 20pt
%\newpage

\end{document}