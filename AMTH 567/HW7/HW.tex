\documentclass[12pt]{report}

\usepackage{amssymb, fullpage, amsmath, esint}
\usepackage{graphicx}

\newtheorem{problem}{Problem}

\newenvironment{solution}[1][\it{Solution}]{\textbf{#1. } }{$\square$}

\graphicspath{ {./} }

\allowdisplaybreaks

\pagestyle{empty}

\def\Z{{\mathbb Z}}
\def\Q{{\mathbb Q}}
\def\C{{\mathbb C}}
\def\R{{\mathbb R}}
\def\N{{\mathbb N}}
\def\eps{{\epsilon}}
\def\O{{\mathcal{O}}}
\newcommand{\floor}[1]{{\left\lfloor#1\right\rfloor}} % Floor function
\newcommand{\ceil}[1]{{\left\lceil#1\right\rceil}} % Ceiling function
\newcommand{\paren}[1]{{\left(#1\right)}} % Parentheses ()
\newcommand{\brac}[1]{{\left\{#1\right\}}} % Curly braces {}
\newcommand{\braces}[1]{{\left[#1\right]}} % Braces []
\newcommand{\abrac}[1]{{\left\langle#1\right\rangle}} % Angle Braces <>
\newcommand{\abs}[1]{{\left|#1\right|}} % Absolute value
\newcommand{\norm}[1]{{\left\|#1\right\|}} % Norm
\newcommand{\eval}[2]{\right|_{#1}^{#2}} % Evaluate

\newcommand{\pp}[2]{\frac{\partial #1}{\partial #2}} % Partial of 1 wrt 2
\newcommand{\ppn}[3]{\frac{\partial^{#1} #2}{\partial #3^{#1}}} % nth Partial of 1 wrt 2
\newcommand{\dd}[2]{\frac{\mathrm{d} #1}{\mathrm{d} #2}} % Partial of 1 wrt 2
\newcommand{\ddn}[3]{\frac{\mathrm{d}^{#1} #2}{\mathrm{d} #3^{#1}}} % nth Partial of 1 wrt 2

\def\ointcc{{\ointctrclockwise}} %counter clockwise contour integral
\def\ointc{{\ointclockwise}} %clockwise contour integral

%dash integral 
\def\Xint#1{\mathchoice
   {\XXint\displaystyle\textstyle{#1}}%
   {\XXint\textstyle\scriptstyle{#1}}%
   {\XXint\scriptstyle\scriptscriptstyle{#1}}%
   {\XXint\scriptscriptstyle\scriptscriptstyle{#1}}%
   \!\int}
\def\XXint#1#2#3{{\setbox0=\hbox{$#1{#2#3}{\int}$}
     \vcenter{\hbox{$#2#3$}}\kern-.5\wd0}}
\def\ddashint{\Xint=}
\def\dashint{\Xint-}


\begin{document}

\large

\begin{center}
 Math 567 Homework 7\\
 Due Soon\\
 By Marvyn Bailly\\
\end{center}

\normalsize

\hrule

%---------------%
%---Problem 1---%
%---------------%

%--status--$

\begin{problem}
    \begin{enumerate}

        \noindent
        \item[a] construct the bilinear transform
        \[ w(z) = \frac{az + b}{cz + d}\]
        that maps the region between the two circles $| z - \frac{1}{4} | = \frac{1}{4}$ and $| z - \frac{1}{2} | = \frac{1}{2}$ into an infinite strip bounded by the vertical lines $u = \mathrm{Re}(w) = 0$ and $u = \mathrm{Re}(w) = 1$. To avoid ambiguity, suppose that the outer circle is mapped to $u = 1$.

        \item[b] Upon finding the appropriate transformation $w$, carefully show that the image of the inner circle under $w$ is the vertical line $u = 0$, and similarly for the outer circle.
    \end{enumerate}
\end{problem}

\begin{solution}
    \noindent
    \begin{enumerate}
        \item[a]
        We wish to construct the bilinear transform
        \[ w(z) = \frac{az + b}{cz + d}\]
        that maps the region between the two circles $| z - \frac{1}{4} | = \frac{1}{4}$ and $| z - \frac{1}{2} | = \frac{1}{2}$ into an infinite strip bounded by the vertical lines $u = \mathrm{Re}(w) = 0$ and $u = \mathrm{Re}(w) = 1$. To avoid ambiguity, suppose that the outer circle is mapped to $u = 1$. To achieve this mapping, we require $z = \frac{1}{2}$ to map to $w = 0$, thus $z_1 = \frac{1}{2}$. We also need $z = 1$ to map to $w = 1$ thus
        \[ A \frac{1 - \frac{1}{2}}{1} = 1 \implies A = 2.\]
        Thus we have that the bilinear transform is given by,
        \[ w(z) = 2 \frac{z - \frac{1}{2}}{z}.\]

        \item[b] 
        Next we wish to verify the transform by showing that the image of the each circle under $w$ is mapped to their corresponding vertical line. First let's consider the inner circle. Let $z = x + i y$ be on the inner circle,
        \[ (x - \frac{1}{4})^2 + y^2 = \frac{1}{16}.\]
        Then,
        \begin{align*}
            w(z) &= 2 \paren{ \frac{(x + iy) - \frac{1}{2}}{(x + iy)}}\\
            &= \paren{ \frac{2(x + iy) - 1}{x + iy} } \paren{ \frac{x - iy}{x - iy}}\\
            &= \frac{2(x^2 + y^2) - (x - iy)}{x^2 + y^2}.
        \end{align*} 
        Next observe that
        \begin{align*}
            \mathrm{Re}(w) &= \frac{2x^2 + 2y^2 - x}{x^2 + y^2}\\
            &= 2 \paren{ \frac{x^2 + y^2 - \frac{1}{2}x }{x^2 + y^2}}\\
            &= \frac{2((x - \frac{1}{4})^2 - \frac{1}{16} + y^2)}{x^2 + y^2}\\
            &= \frac{2(\frac{1}{16} - \frac{1}{16})}{x^2 + y^2}\\
            &= 0.
        \end{align*}
        Thus we see that if $x,y \neq 0$, then $\mathrm{Re}(w(z)) = 0$, otherwise $\mathrm{Re}(w(z)) = z_\infty$ which is the transformation we desired. We can also verify that the inverse transform achieves the desired effect by observing that when $w = ai$
        \[
            z = \frac{1}{2-w} = \frac{1}{2-ai} = \frac{2}{a^2 + 4} + \frac{i a}{a^2 + 4}.
        \]
        Plugging this into the circle equation with $x = \mathrm{Re}(z) = \frac{2}{a^2 + 4}$ and $y = \mathrm{Im}(z) = \frac{a}{a^2 + 4}$ gives
        \begin{align*}
            \frac{1}{16} &= \paren{\frac{2}{a^2 + 4} - \frac{1}{4}}^2 + \paren{\frac{a}{a^2 + 4}}^2\\
            \frac{1}{16} &= - \frac{1}{a^2 + 4} + \frac{4}{(a^2 + 4)^2} + \frac{1}{16} + \frac{a^2}{(a^2 + 4)^2}\\
            0 &= - \frac{1}{a^2 + 4} + \frac{4 + a^2}{(a^2 + 4)^2}\\
            0 &= \frac{-a^2 - 4 + a^2 +4}{(a^2 + 4)^2}\\
            0 &= 0.
        \end{align*} 
        Thus the inverse mapping also holds. 

        \noindent
        Next consider the outer circle. Let $z = x + iy$ be on the circle,
        \[ (x - \frac{1}{2})^2 + y^2 = \frac{1}{4}.\]
        Then,
        \begin{align*}
            w(z) &= 2 \paren{ \frac{(x + iy) - \frac{1}{2}}{(x + iy)}}\\
            &= \paren{ \frac{2(x + iy) - 1}{x + iy} } \paren{ \frac{x - iy}{x - iy}}\\
            &= \frac{2(x^2 + y^2) - (x - iy)}{x^2 + y^2}.
        \end{align*} 
        Next observe that
        \begin{align*}
            \mathrm{Re}(w) &= \frac{2x^2 + 2y^2 - x}{x^2 + y^2}\\
            &= \frac{x^2 + y^2}{x^2 + y^2} + \frac{x^2 + y^2 - x}{x^2 + y^2}\\
            &= 1 + \frac{(x - \frac{1}{2})^2 - \frac{1}{4} + y^2}{x^2 + y^2}\\
            &= 1 + \frac{\frac{1}{4} - \frac{1}{4}}{x^2 + y^2}\\
            &= 1.
        \end{align*}
        Thus we see that if $x,y \neq 0$, then $\mathrm{Re}(w(z)) = 1$, otherwise $\mathrm{Re}(w(z)) = z_\infty$ which is the transformation we desired. We can also verify that the inverse transform achieves the desired effect by observing that when $w = ai + 1$
        \[
            z = \frac{1}{2-w} = \frac{1}{2-ai - 1} = \frac{1}{a^2 + 1} + \frac{i a}{a^2 + 1}.
        \]
        Plugging this into the circle equation with $x = \mathrm{Re}(z) = \frac{1}{a^2 + 1}$ and $y = \mathrm{Im}(z) = \frac{a}{a^2 + 1}$ gives
        \begin{align*}
            \frac{1}{4} &= (\frac{1}{a^2 + 1} - \frac{1}{2})^2 + (\frac{a}{a^2 + 1})^2\\
            \frac{1}{4} &= -\frac{1}{a^2 + 1} + \frac{1}{(a^2 + 1)^2} + \frac{1}{4} + \frac{a^2}{(a^2 + 9)^2}\\
            0 &= \frac{-a^2 - 9}{(a^2 + 9)^2} + \frac{9 + a^2}{(a^2 + 9)^2}\\
            0 &= 0.
        \end{align*} 
        Thus the inverse mapping also holds. 
    \end{enumerate}
\end{solution}

%----------------------------------------------------------------------------------------------------%
%\vskip 20pt
\newpage

%---------------%
%---Problem 2---%
%---------------%

%--status--$

\begin{problem}
    Use the result of Problem 1 to find the steady state temperature $T (x, y)$ in the region bounded by the two circles, where the inner circle is maintained at $T = 0$C and the outer circle at $T = 100$C. Assume $T$ satisfies the two-dimensional Laplace equation.
\end{problem}

\begin{solution}
    
    \noindent
    Building off the previous problem consider the steady state temperature $T (x, y)$ in the region bounded by the two circles, where the inner circle is maintained at $T = 0$C and the outer circle at $T = 100$C. Assume $T$ satisfies the two-dimensional Laplace equation. Thus we have that $T_{xx} + T_{yy} = 0$ with $T_{c_1} = 0$ and $T_{c_2} = 100$. Applying the bilinear transform gives that $T_{uu} + T_{vv} = 0$ with $T(0,v) = 0$ and $T(1,v) = 100.$ Since $T$ has no $v$ dependence, we have that $T_{vv} = 0$. Thus we find that
    \[ T_{uu} = 0 \implies T = cu + d,\]
    and applying the boundary conditions gives
    \[ d = 0 ~~ \text{and} ~~ c = 100. \]
    thus
    \[ T(u,v) = 100 u.\]
    Now to transform it back
    \begin{align*}
        T(w) &= 100 \cdot \mathrm{Re}(w)\\
        &= 100 \cdot \mathrm{Re}\paren{ \frac{2z - 1}{z}}\\
        &= 100 \cdot \mathrm{Re}\paren{\frac{2(x + iy) - 1}{x + iy}}\\
        &= 100 \paren{\frac{2x^2 + 2y^2 - x}{x^2 + y^2}}.
    \end{align*}
\end{solution}

%----------------------------------------------------------------------------------------------------%
%\vskip 20pt
\newpage


\end{document}